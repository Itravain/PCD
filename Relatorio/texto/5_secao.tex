O estudo permite concluir que houve uma redução significativa no tempo de execução do código para entradas relativamente grandes. Para entradas pequenas, o \textit{speedup} foi praticamente nulo, e no caso de 100.000 entradas, o desempenho chegou a piorar. Essa queda pode ser explicada pelo \textit{overhead} gerado pelo OpenMP para gerenciar o paralelismo.

\par O uso do paralelismo pode ser extremamente eficiente para reduzir o tempo de execução de problemas que exigem alto processamento. Entretanto, para problemas de pequena escala, o \textit{overhead} associado pode superar os ganhos, tornando a técnica pouco vantajosa.